\section{Macros}
\section{Notation}
\begin{itemize}
	\item By goal or problem or task we mean the overall task of the thesis, which is develop a machine learning model suitable for predicting prices on the financial market, based on a history of data.
\end{itemize}
\section{Datasets}
Here we describe the various datasets we used for the training and testing of our models. Each description includes the following:
\begin{itemize}
	\item \textbf{Source:} the name and source from where we took the dataset. Including a link or a way to access and download the dataset.
	\item \textbf{Motivation:} a description of what the data in the dataset represent, what their purpose is, how they were gathered and why it is valuable to our research.
	\item \textbf{Size:} a description of how many datapoints the dataset contains, how many features each datapoints has and the overall size of the file.
	\item \textbf{Features:} a description of the features each datapoint of the dataset has and what the features represent.
	\item \textbf{Collection method:} a description of how the data in the dataset was gathered and over what time span.
	\item \textbf{Quality:} a description of the features and drawbacks of the overall dataset e.g. how dependent are the features between themselves.
	\item \textbf{Plot:} a plot of the numeric features of the dataset, its visual representation.
\end{itemize}
\subsection{House sales}
\subsection{Gbpcad}
\subsection{Google}
\section{What metrics we used}
Every metric should be described by what it measures, what its result represents, how valuable it is, when it can be applied.
\section{How we describe models}
Below, in the chapter "Other models" we describe several models we tried to use for the task of predicting prices. The descriptions include the following:
\begin{itemize}
	% \item \textbf{Relevant literature:} 
	\item \textbf{Description:} a short description of how the model works and how it is trained.
	\item \textbf{Motivation:} what the model is usually used for and why we chose to try it out.
	\item \textbf{Features and limitations:} some advantages and benefits of the model, as well as its disadvantages and drawbacks.
	\item \textbf{Parameters:} the description of the parameters of the model and how they affect its training.
	\item \textbf{Metrics:} how we measured the results of the training and testing of the model.
	\item \textbf{Data used:} what combinations of parameters of the model we tested and on what datasets we trained and tested the model.
	\item \textbf{Preprocessing:} how the datasets used were preprocessed for training and testing of the model.
	\item \textbf{Analysis:} an analysis of our results of our training and testing of the model.
	\item \textbf{Picture:} a picture or a plot demonstrating the results obtained from testing the model.
\end{itemize}

\section{Literature review}
Literature review contains:
\begin{itemize}
	\item A list of approaches to the problem.
	\item For each approach, its basic description and its significance to our goal.
	\item Its features and drawbacks compared to our goal.
	\item Its differences, when compared to our goal.
	\item Whether our own results validated the results of the article.
\end{itemize}

