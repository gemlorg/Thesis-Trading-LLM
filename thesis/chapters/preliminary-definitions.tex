Throughout the thesis, by \emph{goal} or \emph{problem} or \emph{task} we mean the overall task of the thesis, which is to develop a machine learning model suitable for predicting prices on the financial market, based on a sequence of observations.

\section{Datasets}
Here we describe the various time series datasets used for the training and testing of the models. In the below table, the columns describe the following:
\begin{itemize}
	\item \textbf{Name:} the name of the dataset.
	\item \textbf{Source:} the source of the data - most was received from our project sponsor, AI Investments. For some a link or a way to access and download the data is provided.
	\item \textbf{Number of data points:} a description of how many data points the dataset contains, how many features each data points has and the overall size of the file.
	\item \textbf{Data points used:} how many of the data points were used in model training (these are taken from the end of the dataset).
	\item \textbf{Time step:} time delta between subsequent data points.
\end{itemize}



\begin{center}
	\begin{tabular}{||p{0.15\linewidth} ||p{0.2\linewidth} | c | p{0.15\linewidth} | c | c ||}
		\hline
		\multicolumn{6}{|c|}{Datasets summary}                                                                                                         \\
		\hline
		Dataset name & Short description                      & Source                           & Number of datapoints & Datapoints used & Time step  \\ [0.5ex]
		\hline\hline
		AAPL         & Apple stock prices                     & Online \cite{apple_source}       & 10943                & 10000           & 1 day      \\
		\hline
		BTCUSD       & Rates of Bitcoin in US Dollars         & AI Investments                   & 40450                & 40450           & 1 hour     \\
		\hline
		EURUSD       & Rates of Euro in US Dollars            & AI Investments                   & 117397               & 117397          & 1 hour     \\
		\hline
		GBPCAD       & Rates  of GB Pound in Canadian Dollars & AI Investments                   & 117423               & 117423          & 1 hour     \\
		\hline
		GBPTRY       & Rates of GB Pound in Turkish Lires     & AI Investments                   & 35965                & 35965           & 1 hour     \\
		\hline
		US500        & US500 stock index                      & AI Investments                   & 118023               & 118023          & 1 hour     \\
		\hline
		Electricity  & Electricity consumption                & Online \cite{electricity_source} & 26304                & 26304           & 15 minutes \\[1ex]
		\hline
	\end{tabular}
\end{center}

\subsection{Explanation of the datasets}
All above except the Electricity dataset represent market time series data - they have the following features:
\begin{itemize}
	\item \textbf{date} - The time at which the data point was recorded.
	\item \textbf{close} - The closing price within the given time interval.
	\item \textbf{high} - The highest price within the given time interval.
	\item \textbf{low} - The lowest price within the given time interval.
	\item \textbf{open} - The opening price within the given time interval.
	\item \textbf{volume} - The volume of the traded asset.
	\item \textbf{adjClose} - The closing price within the given time interval, modified to account for dividends, stock splits, etc., used to better reflect stock value.
\end{itemize}
The target column was the \textbf{close} column.

The Electricity dataset in addition to the \textbf{date} column has 370 other columns, which correspond to different energy consumption clients. Every data point records the energy consumption throughout a constant period for each of those clients. For the target column, we chose the 127th column, since it had the highest coefficient of variation.
This dataset serves to check whether the approach to market time-series generalizes to other types of time-series.

In each experiment, the dataset was divided into 3 parts: the first 70\% of the data were used for the training of the model, then 10\% were used in its validation, and 20\% were used for testing.


\section{Metrics}

\subsection{Mean Squared Error}
The \emph{Mean Squared Error} (MSE) is a metric used in the training of machine learning models, especially in regression tasks. It quantifies how close the predictions of a model are to the actual values.

\subsubsection{Definition}
MSE calculates the average of the squares of the errors. The error is the difference between the values (\(\hat{y}_i\)) predicted by the model and the actual values (\(y_i\)).

\subsubsection{Mathematical Formulation}
The mathematical formula for MSE is given by
\begin{equation}
	MSE = \frac{1}{n} \sum_{i=1}^{n} (y_i - \hat{y}_i)^2,
\end{equation}
where:
\begin{itemize}
	\item \(n\) is the number of predicted datapoints;
	\item \(y_i\) is the actual value for the \(i\)th datapoint;
	\item \(\hat{y}_i\) is the predicted value for the \(i\)th datapoint.
\end{itemize}

\subsection{Accuracy metric}
The \emph{Accuracy}\cite{accuracy} metric measures correctness of class predictions of a model. In the context of time series prediction, it is measured as follows.

When we predict the value \(a_{i+k}\) based on values \([a_{i-n}, a_i]\) in a time series, we
\begin{enumerate}
	\item classify the prediction \(\hat{a}_{i+k}\) into a binary up/down class:
	      \[
		      \mathsf{class}(\hat{a}_{i+k}) =
		      \begin{cases}
			      1, & \text{if } \hat{a}_{i+k} \ge f([a_{i-n}, a_i], k) \\
			      0, & \text{otherwise}
		      \end{cases}
	      \]
	      where in this thesis \(f([a_{i-n}, a_i], k)\) is always \(a_i\), i.e. we classify based on last known data point.
	\item We then classify the value actually observed \(a_{i+k}\) in the same way.
	\item Subsequently, we calculate the fraction of the times when such predictions were correct, i.e. \(\mathsf{class}(\hat{a}_{i+k}) = \mathsf{class}(a_{i+k})\).
	\item This yields an \emph{accuracy} score, between 0 and 1: \(0 \leq \frac{\sum_{i=1}^N[\mathsf{class}(\hat{a}_{i+k}) = \mathsf{class}(a_{i+k})]}{N} \leq 1\). \footnote{Here for \(N\) predictions.}
\end{enumerate}
