In the world of stock markets a major problem is the apparent incalculability of the complex network of factors e.g. how stock prices of one company affect those of another etc. As the environment of stock markets becomes more and more complex, the ability to analise and confidently predict its future becomes of crucial importance for traders, investors and researchers. 

With the recent advent of generative AI and the demonstrable power of Large Language Models a question arises of how these can be used to accurately analise and predict time series market prices in different environments. Therefore, this thesis presents our work on the subject.

First, we look at what work has already been done in the field of LLM time series prediction, in particular what techniques of fine-tuning and input data transformation were used. Then we look at how different, smaller machine learning models deal with time series prediction.

Subsequently, we discuss our methodology; different aplied methods and techniques of input reprogramming, use of prompts and context, and LLM fine-tuning. Next, we present the results we have achieved on the chosen datasets (and compare them to some other known solutions).

Finally, we speculate on the significance of our work, its potential applications in forecasting price time-series.
