\section{Overview}
In the world of stock markets a major problem is the apparent difficulty of calculating the influence of the network factors on one another e.g. how stock prices of one company affect those of another. As the environment of stock markets becomes more and more complex, the ability to analyze and predict its future becomes of crucial importance for traders, investors and researchers.

With the recent advances in the field of generative AI and the demonstrable power of Large Language Models(LLMs) a question arises of if and how these models can be used to accurately analyze and predict time series data and market prices in particular. This thesis presents our work on the subject.

The main technical challenge is the following.

% TODO: reformulate the following paragraph
\textit{Given a sequence of historical observations \(X \in \R^{N\times T}\)
	consisting of \(N\) different 1-dimensional variables across \(T\) time steps, we aim to reprogram a large
	language model \(f(\cdot)\) to accurately forecast the readings at \(H\) succeeding time steps, denoted by \(\hat{Y} \in \R^{N\times H}\),
	with the overall objective to minimize the mean square error between the expected outputs \(Y\) and predictions, i.e., \(\frac1H \sum_{h=1}^H \| \hat{Y}_h - Y_h \|_F^2 \).} \cite{reprogramming_llm}

% \section{Contributions}
\section{Outline} %TODO: correct outline
% First, we look at what work has already been done in the field of LLM time series prediction, in particular what techniques of fine-tuning and input data transformation were used. 
% First, we describe what datasets were used in our experiments and how we measured the results.
First, we provide a detailed description of the datasets used in our experiments and the methodologies used for measuring the results.
Next, we look at how different, simpler machine learning models deal with time series prediction.

% We describe the datasets we used for testing small models and LLMs.

% \section{Contributions}
% \section{Outline}
% First, we look at what work has already been done in the field of LLM time series prediction, in particular what techniques of fine-tuning and input data transformation were used. Then we look at how different, simpler machine learning models deal with time series prediction.

% better be rewriteen, just not a good way of saying it ig. (Heorhii)
We then describe how an LLM works and how it is used for our purposes.

Subsequently, we discuss our own methodology, explaining different methods and techniques we used for input reprogramming, as well as the usage of prompts and context.

Finally, we present the results we have achieved on the chosen datasets and speculate on the significance of our work and its potential applications in forecasting price time-series.
