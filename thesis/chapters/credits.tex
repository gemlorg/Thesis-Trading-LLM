The entire project consisted of various components meticulously coordinated by our team. At the outset, Damian Dąbrowski and Heorhii Lopatin crafted the repository structure and established a set of rules to ensure order and clarity. Assuming the roles of team leads, they efficiently distributed tasks to propel the project forward. The subsequent research phase was a collaborative effort, involving the identification of appropriate training datasets and the analysis of scientific papers relevant to our work, forming the foundation for our final implementation. This phase saw the active participation of all team members.

Following the research, we implemented several simple models, initially as a learning exercise in AI, with each member contributing at least one model. As we approached the second semester, we embarked on developing the main component, the LLM. During this stage, we divided responsibilities based on our competencies. Krzysztof Szostek took charge of writing the thesis, contributing significantly to the document. Ivan Gechu focused on refining simpler models to generate benchmarks, while Heorhii and Damian were responsible for coding and training the LLM.

In the final phase, the entire team collaborated on the thesis, with each member writing sections related to their specific contributions. Our project also included regular consultations with our supervisor, Piotr Hofman, and interactions with our client, AI Investments, led by Mateusz Panasiuk. Initially, the entire team participated in meetings with the client. However, as the project progressed, Damian and Heorhii primarily handled these interactions, discussing LLM-specific aspects, improvement ideas, current results, and challenges. They also delivered the final presentation together. Throughout the project, every team member played a crucial role, ensuring the successful development of our project.

