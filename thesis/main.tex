

% dodaj opcję [licencjacka] dla pracy licencjackiej
% dodaj opcję [en] dla wersji angielskiej (mogą być obie: [licencjacka,en])
\documentclass[en]{pracamgr}

% Dane magistrantów:
\autor{Damian Dąbrowski}{439954}
\autori{Heorhii Lopatin}{456366}
\autorii{Ivan Gechu}{439665}
\autoriii{Krzysztof Szostek}{440011}

\title{Large language models for forecasting market's behaviour}
\titlepl{Duże modele językowe w przewidywaniu giełdy}

%\tytulang{An implementation of a difference blabalizer based on the theory of $\sigma$ -- $\rho$ phetors}

%kierunek: 
% - matematyka, informacyka, ...
% - Mathematics, Computer Science, ...
\kierunek{Computer Science}

% Praca wykonana pod kierunkiem:
% (podać tytuł/stopień imię i nazwisko opiekuna
% Instytut
% ew. Wydział ew. Uczelnia (jeżeli nie MIM UW))
\opiekun{dr Piotr Hofman\\
  Instytut Informatyki\\
  }

% miesiąc i~rok:
\date{May 2024}

%Podać dziedzinę wg klasyfikacji Socrates-Erasmus:
\dziedzina{ 
%11.0 Matematyka, Informatyka:\\ 
%11.1 Matematyka\\ 
% 11.2 Statystyka\\ 
%11.3 Informatyka\\ 
11.4 Sztuczna inteligencja\\ 
%11.5 Nauki aktuarialne\\
%11.9 Inne nauki matematyczne i informatyczne
}

%Klasyfikacja tematyczna wedlug AMS (matematyka) lub ACM (informatyka)
\klasyfikacja{D. Software\\
  D.127. Blabalgorithms\\
  D.127.6. Numerical blabalysis}

% Słowa kluczowe:
\keywords{machine learning, large language models, time series forecasting, market prices}

% Tu jest dobre miejsce na Twoje własne makra i~środowiska:
\newtheorem{defi}{Definicja}[section]

% koniec definicji
\begin{document}
\newcommand{\rozdzial}[2]{
	\chapter{#1}
	\input{chapters/#2}
}

\maketitle

%tu idzie streszczenie na strone poczatkowa
\begin{abstract}
        This thesis concerns research into the use of machine learning 
        and large language models in market analysis, focusing on market
        predictions.

\end{abstract}

\tableofcontents
%\listoffigures
%\listoftables

\rozdzial{Introduction}{introduction}
\rozdzial{Related work}{related-work}
\rozdzial{Other models}{other-models}
\rozdzial{Methodology }{methodology}
\rozdzial{Main results}{results}
\rozdzial{Forecasting applications}{forecasting-application}
\rozdzial{Conclusion}{conclusion}
\appendix
\rozdzial{Visualisation}{visualisation}
%here appendix
\begin{thebibliography}{99}
	\addcontentsline{toc}{chapter}{Bibliografia}

	\bibitem[Bea65]{beaman} Juliusz Beaman, \textit{Morbidity of the Joll function}, Mathematica Absurdica, 117 (1965) 338--9.

	\bibitem[Blar16]{eb1} Elizjusz Blarbarucki, \textit{O pewnych
		aspektach pewnych aspektów}, Astrolog Polski, Zeszyt 16, Warszawa
	1916.

\end{thebibliography}

\end{document}


%%% Local Variables:
%%% mode: latex
%%% TeX-master: t
%%% coding: latin-2
%%% End:
