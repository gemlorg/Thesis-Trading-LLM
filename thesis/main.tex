% dodaj opcję [licencjacka] dla pracy licencjackiej
% dodaj opcję [en] dla wersji angielskiej (mogą być obie: [licencjacka,en])
\documentclass[licencjacka, en]{pracamgr}

% Dane magistrantów:
\autor{Damian Dąbrowski}{439954}
\autori{Heorhii Lopatin}{456366}
\autorii{Ivan Gechu}{439665}
\autoriii{Krzysztof Szostek}{440011}

\title{Large language models for forecasting market behaviour}
\titlepl{Duże modele językowe w przewidywaniu zmian rynkowych}

\usepackage{hyperref}
\usepackage{graphicx}
\usepackage{placeins}
\usepackage{subcaption}
\usepackage{amssymb}
\usepackage{amsmath}
\usepackage{xurl}
%\tytulang{An implementation of a difference blabalizer based on the theory of $\sigma$ -- $\rho$ phetors}

%kierunek: 
% - matematyka, informatyka, ...
% - Mathematics, Computer Science, ...
\kierunek{Computer Science}

% Praca wykonana pod kierunkiem:
% (podać tytuł/stopień imię i nazwisko opiekuna
% Instytut
% ew. Wydział ew. Uczelnia (jeżeli nie MIM UW))
\opiekun{dr Piotr Hofman\\
  Institute of Informatics\\
  }

% miesiąc i~rok:
\date{May 2024}

%Podać dziedzinę wg klasyfikacji Socrates-Erasmus:
\dziedzina{ 
%11.0 Matematyka, Informatyka:\\ 
%11.1 Matematyka\\ 
% 11.2 Statystyka\\ 
%11.3 Informatyka\\ 
11.4 Artificial Intelligence\\ 
%11.5 Nauki aktuarialne\\
%11.9 Inne nauki matematyczne i informatyczne
}

%Klasyfikacja tematyczna wedlug AMS (matematyka) lub ACM (informatyka)
\klasyfikacja{Computing methodologies$\,\to\,$Neural networks}
% Full:
% Computing methodologies
% Machine learning
% Machine learning approaches
% Neural networks


% Słowa kluczowe:
\keywords{machine learning, large language models, time series forecasting, market prices}

% Tu jest dobre miejsce na Twoje własne makra i~środowiska:
\newtheorem{defi}{Definicja}[section]
\newcommand{\R}{\mathbb{R}}

% koniec definicji
\begin{document}
\newcommand{\rozdzial}[2]{
	\chapter{#1}
	\label{chap:#2}
	\input{chapters/#2}
}

\maketitle

%tu idzie streszczenie na strone poczatkowa
\begin{abstract}
	This thesis concerns research into the use of machine learning
	and large language models in market analysis, focusing on market
	predictions.
\end{abstract}

\tableofcontents
%\listoffigures
%\listoftables

\rozdzial{Introduction}{introduction}
\rozdzial{Preliminary definitions \& guidelines}{preliminary-definitions}
% \rozdzial{Literature review}{literature}
% \rozdzial{Related work}{related-work}
\rozdzial{Other models}{other-models}
\rozdzial{Large Language Model}{large-language-model}
% \rozdzial{Methodology }{methodology}
\rozdzial{Main results}{results}
% \rozdzial{Forecasting applications}{forecasting-application}
\rozdzial{Conclusion}{conclusion}
% \appendix
% \rozdzial{Visualisation}{visualisation}
%here appendix
\rozdzial{Credits}{credits}
% \chapter*{Credits}
% \addcontentsline{toc}{chapter}{Credits}
% The entire project consisted of various components meticulously coordinated by our team. The research phase was a collaborative effort, involving the identification of appropriate training datasets and the analysis of scientific papers relevant to our work, forming the foundation for our final implementation. This phase saw the active participation of all team members.

Following the research, we implemented several simple models, initially as a learning exercise in AI, with each member contributing at least one model. As we approached the second semester, we embarked on developing the main component, the LLM. During this stage, we divided responsibilities based on our competencies. Krzysztof Szostek took charge of writing the thesis, contributing significantly to the document. Ivan Gechu focused on refining simpler models to generate benchmarks, while Heorhii Lopatin and Damian Dąbrowski were responsible for coding and training the LLM.

In the final phase, the entire team collaborated on the thesis, with each member writing sections related to their specific contributions. Our project also included regular consultations with our supervisor, Piotr Hofman, and interactions with our client, AI Investments, led by Mateusz Panasiuk. Initially, the entire team participated in meetings with the client. However, as the project progressed, Damian and Heorhii primarily handled these interactions, discussing LLM-specific aspects, improvement ideas, current results, and challenges. Throughout the project, every team member played a crucial role, ensuring the successful development of our project.

ChatGPT was used during the process of writing and editing of the thesis.


\begin{thebibliography}{99}
	\addcontentsline{toc}{chapter}{Bibliography}


	\bibitem{apple_source} \url{https://finance.yahoo.com/quote/AAPL/history/}

	\bibitem{electricity_source} \url{https://archive.ics.uci.edu/dataset/321/electricityloaddiagrams20112014}

	\bibitem{overfitting} \url{https://en.wikipedia.org/wiki/Overfitting}

	%	\bibitem{random_forest} Scornet, E., Biau, G., \& Vert, J. P. (2015). Consistency of Random Forests. arXiv preprint arXiv:1511.05741.
	%
	%	\bibitem{logistic_regression} Meng, L., Cao, J., Zhang, C., Yu, S., \& Yang, Q. (2020). Sufficient Dimension Reduction for Logistic Regression. arXiv preprint arXiv:2008.13567.

	\bibitem{linear_regression} Kuchibhotla, A. K., Brown, L. D., Buja, A. \& Cai, J. (2019). All of linear regression. arXiv preprint arXiv:1910.06386.

	\bibitem{support_vector_machine} Steinwart, I., \& Christmann, A. (2006). Estimating conditional quantiles with the help of the pinball loss. arXiv preprint arXiv:math/0612817.

	% TODO
	\bibitem{multilayer_perceptron} (Add complete citation when available).

	\bibitem{convolutional_neural_network} Simonyan, K., \& Zisserman, A. (2015). Very Deep Convolutional Networks for Large-Scale Image Recognition. arXiv preprint arXiv:1511.08458.

	\bibitem{cnn_diagram_source} \url{https://stanford.edu/~shervine/teaching/cs-230/cheatsheet-convolutional-neural-networks}

	\bibitem{residual_neural_network} He, K., Zhang, X., Ren, S., \& Sun, J. (2015). Deep Residual Learning for Image Recognition. arXiv preprint arXiv:1512.03385.

	\bibitem{token_embeddings} Jurafsky, D., \& Martin, J. H. (n.d.). Token Embeddings. Retrieved from \url{https://web.stanford.edu/~jurafsky/slp3/6.pdf}.

	\bibitem{language_models} Li, Z., Li, J., \& Liu, X. (2023). Efficient Language Models with Dynamic Token Dropping. arXiv preprint arXiv:2303.18223.

	\bibitem{llmintro} Zhang, Z., Li, X., \& Yang, W. (2023). An Introduction to Large Language Models. arXiv preprint arXiv:2304.00612.

	\bibitem{attention_is_all_you_need} Vaswani, A., Shazeer, N., Parmar, N., Uszkoreit, J., Jones, L., Gomez, A. N., Kaiser, L., \& Polosukhin, I. (2017). Attention is All You Need. arXiv preprint arXiv:1706.03762.

	\bibitem{llama} Touvron, H., Lavril, T., Izacard, G., Martinet, X., Lachaux, M. A., Lacroix, T., ... \& Jegou, H. (2023). LLaMA: Open and Efficient Foundation Language Models. arXiv preprint arXiv:2302.13971.

	\bibitem{llama_code} Meta AI. (2023). LLaMA GitHub Repository. Retrieved from \url{https://github.com/meta-llama/llama}.

	\bibitem{llama2} Touvron, H., Martin, X., Stone, A., Albert, P., Almahairi, A., Babaei, Y., ... \& Jegou, H. (2023). LLaMA 2: Open Foundation and Fine-Tuned Chat Models. arXiv preprint arXiv:2307.09288.

	\bibitem{reprogramming_llm} Wang, Y., Xu, J., \& Lin, J. (2023). Reprogramming Large Language Models with Synthetic Data. arXiv preprint arXiv:2310.01728.

	\bibitem{nie_et_al} Nie, Y., Nguyen, N. H., Sinthong, P., \& Kalagnanam, J. (2023). A time series is worth 64 words: Long-term forecasting with transformers. In International Conference on Learning Representations.

\end{thebibliography}

%\rozdzial{Bibliography}{bibliography}
\end{document}


%%% Local Variables:
%%% mode: latex
%%% TeX-master: t
%%% coding: latin-2
%%% End:
